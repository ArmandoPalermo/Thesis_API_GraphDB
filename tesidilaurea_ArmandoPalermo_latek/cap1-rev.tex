Negli ultimi anni le cryptovalute hanno attirato moltissima attenzione. Ad oggi si contano migliaia di valute virtuali diverse, tra cui le più note Bitcoin, Ethereum e Tether. La prima valuta virtuale, il Bitcoin, è stata introdotta nel 2008, quando fu pubblicato un link ad un articolo intitolato “Bitcoin: A Peer-to-Peer Electronic Cash System”. L'autore di tale documento è ancora sconosciuto, nell'articolo viene specificato infatti l'alias Satoshi Nakamoto, ma ad oggi nessuno conosce chi realmente si nasconda dietro questo pseudonimo. Di certo, il Bitcoin è un progetto rivoluzionario per quanto concerne i sistemi di pagamento, in quanto è in grado di garantire sicurezza, irreversibilità e privacy delle transazioni in modo \emph{decentralizzato}, cioè senza delegare tali compiti ad uno o pochi intermediari fidati.

Uno degli aspetti che contraddistinguono il sistema Bitcoin, e le criptovalute in generale, è l'utilizzo di un registro digitale condiviso tra un network di dispositivi, dove sono tracciati tutti i trasferimenti di Bitcoin (le cosiddette \emph{transazioni}) in modo immutabile e decentralizzato. Tale  registro prende il nome dall'unione dei termini utilizzati  nell'articolo seminale di Nakamoto per descrivere la struttura (composta da una catena di blocchi) nella quale venivano memorizzati i dati \cite{bitcoin-paper}, ovvero block e chain. Oggigiorno il termine blockchain ha assunto un significato più ampio, tipicamente si riferisce al registro, alla rete di dispositivi e soprattutto al protocollo per la gestione decentralizzata delle transazioni. 

Purtroppo, la possibilità di effettuare transazioni anonime transnazionali a costi contenuti ha reso le criptovalute, e quindi la tecnologia blockchain, particolarmente appetibili ad utenti malevoli. Molte organizzazioni criminali utilizzano infatti le criptovalute per scopi illeciti, come ad esempio il riciclaggio di denaro o il pagamento di armi e droga.
Pertanto l'individuazione di movimenti e pattern di transazioni sospetti risulta di fondamentale importanza per la catalogazione e la prevenzione di tali comportamenti.
Tale attività deve necessariamente avvalersi di sistemi software a supporto degli analisti, vista l'enorme mole di dati memorizzati all'interno di una blockchain, che al momento, per quanto riguarda Bitcoin, ammonta a circa 590 Gigabyte \cite{blockchair}. Un elemento essenziale per l'analisi è la possibilità di rappresentare graficamente gli schemi di relazioni sospette tramite opportuni strumenti di visualizzazione. Il tipo di visualizzazione deve essere progettato per essere funzionale al tipo di analisi che si desidera effettuare. Ad oggi, sono stati sviluppati diversi modelli di visualizzazione. 

Ad esempio \textit{BitExtract}, il quale attraverso dettagli tecnici permette di avere informazioni riguardo l'evoluzione delle transazioni e le relazioni tra i vari exchange di criptovalute.
Oppure \textit{Etherviewer}, un tool grafico che pone l'attenzione sugli smart contract della blockchain di Ethereum, pensato per essere utilizzato anche da persone senza competenze tecniche, permette di estrarre dati in tempo reale dalla catena di blocchi permettendo agli utenti di interagire con essi \cite{vis-tools-overview}. 

La maggior parte delle volte, questi tipi di strumenti prendono in considerazione dei dati in tempo reale ottenuti direttamente dalla blockchain di riferimento.
In alcuni casi è necessario tener traccia di informazioni passate, molte volte tramite l'utilizzo di database che siano in grado di permettere la navigazione dei dati sulla blockchain in modo efficace ed efficiente.


Lo scopo di questo progetto di tesi è lo sviluppo di un database a grafo che permetta di ottenere in modo efficiente specifici dati della blockchain Bitcoin da dare in pasto ad un sistema di analisi visuale precedentemente realizzato \cite{TesiBITVAS}. Inoltre, per permettere un'adeguata interazione tra quest'ultimo e il database il progetto include anche lo sviluppo di un sistema di \emph{API RESTful}. La tesi sarà strutturata come segue:


\begin{itemize}
    \item Capitolo 2 - \textbf{Concetti preliminari}: Verranno approfonditi i concetti fondamentali per permettere una comprensione adeguata degli argomenti trattati nella tesi.

    \item Capitolo 3 - \textbf{Tecnologie utilizzate}: Verranno elencate e spiegate le tecnologie che sono state utilizzate l'implementazione del database a grafo e della API RESTful.
    
    \item Capitolo 4 - \textbf{Obiettivo del progetto} : Verrà formalizzato l'obiettivo del progetto, con un focus sulle specifiche dei sistemi da collegare scelti come riferimento.

    \item Capitolo 5 - \textbf{Progetto ed implementazione dell'API}: Verranno presentate le implementazioni dei due sistemi sviluppati, con particolare attenzione alla struttura della base di dati e all'architettura dell'API.

    \item Capitolo 6 - \textbf{Test e validazione API} : Verrà mostrata la fase di test della API  sfruttando l'utilizzo di una documentazione sviluppata tramite Swagger.


    \item Capitolo 7 - \textbf{Conclusioni}: Verranno riassunti gli obiettivi raggiunti e possibili sviluppi futuri.

\thispagestyle{mystyle}
\end{itemize}