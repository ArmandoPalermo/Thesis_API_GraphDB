In questo capitolo vengono introdotti alcuni concetti che permettono al lettore di acquisire le nozioni fondamentali per un'adeguata fruizione del materiale presente nei capitoli successivi.


\subsection{RESTful API}
In senso più ampio, un'API (o anche Application Programming Interface) è un insieme di protocolli attraverso i quali viene permessa l'interazione di più  applicativi.
Dunque è considerata come un software di intermezzo  che mette in comunicazione  un fornitore di dati e l'utente destinatario di questi ultimi.

Una \emph{RESTful API} \cite{RTF}  è un'interfaccia di programmazione che segue i vincoli architetturali REST (anche detti Representational State Transfer), i quali possono essere implementati in diversi modi anche a seconda delle esigenze.
Per essere considerata RESTful, un’API deve rispettare determinati criteri.
La struttura del sistema deve comprendere un client ed un server che scambiano informazioni tramite \textit{HTTP} (HyperText Transfer Protocol). Questa separazione permette un'evoluzione indipendente di entrambe le entità, aumentando portabilità e scalabilità del sistema.

Inoltre, la comunicazione deve essere \emph{stateless},  in cui ogni richiesta da parte del client deve contenere tutte le informazioni necessarie al server per comprenderla e soddisfarla. Il server non deve essere in grado di sfruttare le informazioni di contesto legate a richieste precedenti dei client.


I dati che vengono scambiati in una sessione devono essere trasferiti in una forma standard e completa, in modo tale che le risorse richieste siano correttamente identificabili e manipolabili dal client.
Il sistema, strutturato adeguatamente, deve essere in grado di bilanciare il carico delle richieste in entrata.
Anche attraverso opzioni di \textit{caching}, che permettono ai client di riutilizzare gli stessi dati per una richiesta equivalente ad un'altra già effettuata in precedenza.

Ogni risorsa presente sul server, che può essere richiesta ad una API RESTful deve essere identificata tramite un \emph{endpoint}, ovvero un \emph{URI}( o anche Uniform Resource Identifier) al quale il client può  richiedere l'entità ad esso associata.
Le informazioni scambiate dalle API di cui sopra possono avere diversi formati, tra cui il più utilizzato è il \emph{JSON}, in quanto più diffuso e di facile interpretazione.
Di fondamentale importanza è la possibilità di utilizzare delle intestazioni in grado di personalizzare la API, permettendo agli sviluppatori di gestire parametri cruciali come le autorizzazioni per l'accesso al server, i cookie, gli URI o anche il caching delle risorse. Esistono intestazioni sia nella richiesta, e quindi lato client, sia in risposta da parte del server, con i corrispondenti codici di stato e informazioni riguardo la connessione HTTP.




\subsection{Blockchain e Bitcoin}
La blockchain può essere definita come un registro  decentralizzato e immutabile, in grado di facilitare la registrazione e il tracciamento delle transazioni o degli asset, tangibili o non, all'interno di una rete \cite{blockchain-ibm}.
Più precisamente è una struttura dati, in continua crescita, le cui unità fondamentali, collegate tramite strumenti crittografici, sono dette  \emph{blocchi}.
L'unione di questi blocchi, formano appunto una catena, che dà il nome alla struttura.
Ognuno di essi è collegato al precedente tramite un riferimento crittografico strettamente dipendente dal contenuto del blocco stesso. Ciò rende la blockchain immutabile, in quanto anche un piccolissimo cambiamento all'interno delle informazioni contenute nei blocchi, porterebbe inevitabilmente ad una variazione del riferimento 
stesso. 

La tecnologia con la quale viene identificata la blockchain è denominata \emph{DLT, Distributed Ledger Technology} (Tecnologia di Registro Decentralizzata), in grado di facilitare l'immagazzinamento dei dati, attraverso un insieme di protocolli, in grado di permettere l'aggiornamento dei record ad utenti  facenti parte della rete, anche da luoghi diversi \cite{blockchain-DLT-digGovernment}.
Questo è possibile in quanto ogni utente possiede una copia locale dell'intero registro, tramite la quale i partecipanti che possiedono determinate autorizzazioni e informazioni, possono essere in grado di tracciare un record passato già validato.

Nonostante la natura decentralizzata del sistema, i dati delle transazioni vengono registrati in maniera sincrona da tutti gli utenti facenti parte della rete. 
\thispagestyle{mystyle}
Una transazione è considerata l'evento di base che può essere analizzato nella blockchain, e non altro che un'interazione tra due account della rete.
Il classico esempio di transazione è lo scambio  di cryptovaluta tra due utenti.
Per evitare che un determinato pool di utenza ottenga l'egemonia sulla validazione delle transazioni, viene utilizzato un meccanismo di consenso denominato \textit{Proof-Of-Work}.

La Proof-of-Work (PoW) \cite{bitcoin-paper}, è un algoritmo di validazione, che consiste nella soluzione di \textit{crypto-puzzle}, al fine di individuare un hash (in formato \textit{SHA-256}) con determinate caratteristiche definite dai protocolli della rete.
Per la risoluzione di questi tipi di enigmi è richiesta una grande potenza computazionale, la quale viene messa a disposizione da parte dei \textit{miner}. Nel caso in cui un qualsiasi nodo della rete riesca a trovare una soluzione, e quindi l'hash del blocco che in corso di validazione, esso viene comunicato agli altri partecipanti alla rete.
In questa maniera, il blocco e le relative transazioni vengono registrate su tutte le copie locali, sancendo quindi il passaggio alla risoluzione del crypto-puzzle utile alla validazione del blocco successivo.


\subsection{Graph Database}
 Recentemente la scelta dei database relazionali è la più comune quando si tratta di immagazzinare dei dati, i quali sono molto efficienti, ammenochè non si abbia a che fare con una grande quantità di relazioni che legano tali informazioni.
 In questo caso le operazioni di \emph{join}, potrebbero risultare computazionalmente molto dispendiose.
 Questo ha portato inevitabilmente ad una imponente ricerca di soluzioni alternative, tra cui le basi di dati a grafo, in grado di garantire una manipolazione più efficace ed efficiente delle informazioni strutturate in modo particolare.
 
Il concetto di base di dati \cite{GraphDatabase-Survey} modellata come un grafo viene utilizzato la prima volta durante gli inizi degli anni novanta, il cui interesse diminuì nel periodo successivo, data la crescita e l'attenzione verso altri tipi di modellazione.
Negli ultimi anni, si ha sempre più la necessità di memorizzare una moltitudine di dati eterogenei fra loro, ma allo stesso tempo collegati.
Questo è uno dei motivi fondamentali del ravvivamento dell'attenzione verso questo tipo di modellazione,
infatti i  database a grafo vengono utilizzati in contesti nei quali le relazioni tra le informazioni sono più importanti dei dati stessi.

La struttura dei dati assume la forma di un grafo etichettato diretto 
 e prevede l'dentificazione delle entità della realtà da rappresentare,  in nodi capaci di contenere tutte le informazioni ad esse associate, le cui relazioni vengono definite da archi che le collegano   \cite{Graph-modelling-1990}.
 In questo modo la comprensione e l'analisi dei dati risulta più immediata. Nonostante lo spazio di  memorizzazione della stessa quantità di dati sia minore nei database relazionali, l'estrapolazione di essi, all'aumentare della dimensione con tutte le relative relazioni, tende ad essere  più efficiente nei database a grafo \cite{Graph-comparison-relational}.
 
 Questo tipo di struttura è molto utile alla rappresentazione di realtà che devono tenere in considerazione grandi moli di dati, come ad esempio la blockchain di Bitcoin, la quale viene sempre modellata come una rete. Ciò rende le basi di dati a grafo, una delle scelte migliori per la sua modellazione.
 Il paradigma di rappresentazione più utilizzato è quello legato alle transazioni che possono essere modellate in nodi appartrenenti al database, con tutti i relativi atttributi, le cui relazioni esistono solo nel caso in cui un output di una tansazione venga utilizzato in input ad un'altra, andando così a modellare la base di dati come una sorta di grafo delle transazioni.
 \thispagestyle{mystyle}







