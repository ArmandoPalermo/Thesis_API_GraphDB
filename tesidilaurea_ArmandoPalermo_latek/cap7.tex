
Con l'imponente affermazione delle blockchain si è  sviluppata l'esigenza di un sistema di archiviazione adeguato per i dati contenuti al suo interno.
Questo studio pone il focus sulla progettazione e l'implementazione di un database a grafo e di un'API in grado di estrarre le informazioni presenti nella blockchain di Bitcoin, in modo efficace ed efficiente.

La struttura della base di dati, basata su una clusterizzazione temporale dei dati, permette un'estrazione più immediata e intuitiva dei flussi di cryptovaluta che riguardano un insieme di blocchi, senza avere necessità di scendere al livello dei blocchi o delle transazioni, nella struttura ad albero formata.

Inoltre, l'implementazione di un sistema API per l'analisi di tali dati permette l'interazione di essi con altri applicativi con obiettivi diversi. Lo sviluppo di moduli indipendenti per la gestione delle query e la generazione dei file JSON  garantisce scalabilità, manutenibilità e facile comprensione del codice sorgente.
Questo approccio modulare non solo facilita l’aggiornamento del sistema, ma offre anche una maggiore flessibilità nell’analisi dei dati, permettendo di eseguire query specifiche e ottenere risposte mirate per diverse esigenze analitiche.

I test effettuati hanno confermato il corretto funzionamento del sistema. dimostrando la sua efficacia nel reperimento dei dati tramite una documentazione generata dai tool messi a disposizione da Swagger.
Inoltre, si  dimostrata efficace anche nell'integrazione con BITVAS, riuscendo ad estrapolare i dati necessari per la CombinedVisualization e la MinerVisualization.

Al momento i soli punti di accesso messi a disposizione dall'API sono quelli utili al reperimento di dati attraverso richieste GET infatti, ulteriori miglioramenti per lo sviluppo dell'API potrebbero includere l'aggiunta di altri endpoint utili all'inserimento o aggiornamento dei dati presenti all'interno del database.

I dati vengono inseriti manualmente all'interno della base di dati, ciò potrebbe essere poco adatto ad utenti poco esperti che fanno utilizzo di tale sistema, quindi potrebbe essere necessario uno script che prelevi in modo automatico i dati più recenti dalla blockchain e li immetta nel database tramite i suddetti endpoint.


